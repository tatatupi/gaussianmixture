\documentclass[12pt,reqno,a4paper,oneside]{article}

\usepackage{amsfonts, amsmath,
	        amssymb, amsthm}  % Fontes e símbolos matemáticos da AMS.
\usepackage{graphicx}         % Importação de gráficos.
\usepackage{subfig}
\usepackage[brazil]{babel}    % Idioma português (Brasil) como padrão.
\usepackage[utf8]{inputenc}   % Codificação UTF-8.
%\usepackage{natbib}           % Para citações.
%\usepackage{cite}             % Idem.

\usepackage{bm}          % Equações em negrito com qualidade superior.
\usepackage{booktabs}    % Edição de linhas de tabelas.
\usepackage{caption}     % Edição do texto dos títulos de tabelas.
\usepackage{color}       % Edição de cores.
\usepackage{indentfirst} % Indentará sempre o 1º parágrafo de cada seção do relatório.
%\usepackage{enumitem}   % Edição mais elaborada de listas
                         % Conflito com ambiente "enumerate" no beamer!.
%\usepackage{hyperref}   % Criação de "hyperlinks".
\usepackage{multirow}    % Para obter uma célula com uma coluna e várias linhas.
\usepackage{ragged2e}    % Opções de alinhamento do texto.

%\usepackage[FIGTOPCAP]{subfigure} % Acesso a figuras dentro de pastas.

\usepackage[top=2cm, bottom=2cm,
            left=2cm, right=2cm]{geometry} % Margens do texto.

\captionsetup{justification = centering} % Centraliza os títulos das tabelas.

% Símbolo matemático para independência entre variáveis aleatórias:

\newcommand\indep{\protect\mathpalette{\protect\independenT}{\perp}}\def\independenT#1#2{\mathrel{\rlap{$#1#2$}\mkern2mu{#1#2}}}

\begin{document}

\title{\huge Relatório do Trabalho de\\
	Estatística Bayesiana I}
\author{\\
	\\
	\huge Mistura de Normais\\
	\huge com Variância Contaminada\\
	\\
	\\
	\\
	\\
	\Large Caio Balieiro\\
	\Large Taiguara Melo Tupinambás\\
	\Large Walmir dos Reis Miranda Filho\\
	\\
	\\
	\\
	\\
	\Large Prof. Dani Gamerman\\
	\Large Profª Rosangela Helena Loschi\\
	\\
	\\
	\\
	\\
	\\
	Programa de Pós-Graduação em Estatística\\
	Instituto de Ciências Exatas\\
	Universidade Federal de Minas Gerais\\
	\\
	\\
	\\
	\\
	\\
	\\}
\date{Belo Horizonte, 2 de dezembro de 2019}
\maketitle

\newpage

\section{Introdução}

O presente trabalho tem como objetivo obter, dada uma densidade \textit{a posteriori} conjunta dos parâmetros de um modelo probabilístico para uma amostra previamente observada, as densidades \textit{a posteriori} marginais de cada parâmetro, bem como as estatísticas de média; variância; assimetria e curtose associadas, a partir da implementação de três métodos numéricos, a saber: (i) integração via quadratura de Riemann; (ii) reamostragem por importância (em inglês, \textit{Sample Importance Resampling}, ou SIR); e (iii) integração via Monte Carlo em cadeias de Markov (em inglês, \textit{Markov Chain Monte Carlo}, ou MCMC) com inovações dadas pelo algoritmo de Metropolis-Hastings (MH).

Para o modelo que gerou a amostra observada, será pressuposto que sua especificação é dada por uma mistura finita de normais. Sejam $X_{1}, \ldots, X_{n}$ amostras aleatórias independentes, condicionalmente a um vetor de parâmetros $\bm{\theta} = (\mu, \sigma^2, \nu)$, e identicamente distribuídas com função densidade dada por
\begin{equation}\label{eq:dist_am}
f(x | \mu, \sigma^2, \nu) = \nu \phi(x | \mu, 100 \sigma^2) + (1 - \nu) \phi(x | \mu, \sigma^2), \ x \in \mathbb{R},
\end{equation}
\noindent onde $\phi(x | \mu, \sigma^2) = (2\pi\sigma^2)^{-1} \exp[-(x - \mu)^2/(2\sigma^2)]$ denota a função densidade da distribuição normal com média $\mu$ e variância $\sigma^2$ avaliada no ponto $x$. Para o suporte de cada parâmetro, tem-se que $\mu \in \mathbb{R}, \sigma^2 \in \mathbb{R}_+$ e $\nu \in (0,1)$.

A mistura finita descrita em \eqref{eq:dist_am} possui duas componentes normais que diferem nas respectivas variâncias: para a primeira componente, o valor é 100 vezes o da segunda. Desta forma, tem-se uma mistura finita de normais com variância \textit{contaminada}.

Para os parâmetros $\mu$, $\sigma^2$ e $\nu$, será pressuposto que cada um segue uma distribuição \textit{a priori} predeterminada: $\mu | \sigma^2 \sim{N} (m, V \sigma^2)$, onde $N(\cdot)$ denota a distribuição normal com média $m \in \mathbb{R}$ e variância $V \sigma^2$, $V > 0$; $\sigma^2 \sim{GI} (a,d)$, onde $GI(\cdot)$ denota a distribuição gama inversa com parâmetros de forma $a > 0$ e de taxa $d > 0$ (inverso da escala); e $\nu \sim{U}(0,1)$, a distribuição uniforme contínua padrão.

Para gerar uma amostra aleatória do modelo em $\eqref{eq:dist_am}$, foi utilizada uma representação hierárquica (Lachos \textit{et al.}, 2013)\cite{Lachos2013} tal que
\begin{equation}
X_i | \mu, \sigma^2, U_{i} = u_i \sim{N}(\mu, \sigma^2 u_i^{-1}), \quad U_i | \mu \sim{p_d}(1,100) : P(U_i = 100) = \nu. \label{eq:hier}
\end{equation}
\noindent onde $p_d(a,b)$ denota uma função de probabilidade (discreta) que atribui massa probabilística apenas aos pontos $a$ e $b$.

A partir do produto entre a função de verossimilhança $f(\bm{x} | \mu, \sigma^2, \nu)$ para uma amostra $\bm{x} = (x_1, \ldots, x_n)$ gerada através da representação hierárquica em \eqref{eq:hier} e das densidades \textit{a priori} para cada parâmetro do modelo, obtém-se o núcleo (em inglês, \textit{kernel}) da densidade \textit{a posteriori} conjunta $p(\mu, \sigma^2, \nu | \bm{x})$. Como a expressão da densidade \textit{a posteriori} possui uma constante de proporcionalidade $f(x_1, \ldots, x_n) = f(\bm{x})$ não facilmente calculável, mas que não depende dos parâmetros, pode-se usar o núcleo daquela para inferir sobre cada um dos parâmetros. No modelo de mistura finita de duas componentes normais com variância contaminada, temos que
\begin{align}
p(\mu, \sigma^2, \nu | \bm{x})
&= \dfrac{f(\bm{x} | \mu, \sigma^2, \nu) \times p(\mu, \sigma^2, \nu)}{f(\bm{x})} \propto \prod_{i=1}^{n} f(x_i) \times p(\mu | \sigma^2) \times p(\sigma^2) \times p(\nu) \nonumber\\
&\propto \prod_{i=1}^{n} \left[ \nu \phi(x_i | \mu, 100 \sigma^2) + (1 - \nu) \phi(x_i | \mu, \sigma^2) \right] \times \nonumber \\
&\times \phi(\mu | m, V \sigma^2) \times \dfrac{d^a}{\Gamma(a)} \left(\dfrac{1}{\sigma^2}\right)^{a + 1} \exp\left(-\dfrac{d}{\sigma^2}\right) \nonumber \\
&\propto \left(\dfrac{1}{\sigma^2}\right)^{(n + 1)/2 + a + 1} \exp\left\{-\dfrac{\left[(\mu - m)^2 / (2V) + d\right]}{\sigma^2}\right\} \times \textrm{A}(\bm{x} | \mu, \sigma^2, \nu), \label{eq:dist_post}
\end{align}
onde
\begin{equation*}
\textrm{A}(\bm{x} | \mu, \sigma^2, \nu) = \prod_{i=1}^{n} \left\{  \dfrac{\nu}{10} \exp\left[-\dfrac{(x_i - \mu)^2}{200\sigma^2}\right] + (1 - \nu) \exp\left[-\dfrac{(x_i - \mu)^2}{2\sigma^2}\right] \right\}.
\end{equation*}

Evidentemente, trabalhar diretamente com o lado direito em \eqref{eq:dist_post} não é computacionalmente agradável, uma vez que este é um produto de várias quantidades sempre menores do que 1, muitas delas próximas de 0, o que pode levar a problemas de aproximação numérica. Desta forma, sempre que for necessário calcular o núcleo da \textit{posteriori}, dada uma amostra, inicialmente será tomada a soma do logaritmo de todos as quantidades que compõem este produto e só depois esta mesma soma será exponenciada. Dependendo do quão negativa for a soma (o logaritmo de termos entre 0 e 1 é sempre negativo), quando exponenciada ela pode ou não ser menor que o mínimo aceito pelo \textit{R}, se tornando igual a 0.

Logo, na geração da amostra $\bm{x}$ deve-se escolher valores de $\mu$, $\sigma^2$ e $\nu$ tais que o núcleo de $p(\mu, \sigma^2, \nu | \bm{x})$ não seja numericamente nulo para os verdadeiros valores dos parâmetros. Considerando o uso da linguagem de programação \textit{R} (R Core Team, 2019)\cite{RCoreTeam2019}, esta escolha não é difícil. Na versão 3.5.3, o menor valor inteiro que pode ser tomado como argumento da função \verb|exp()| do \textit{R} é $-745$. Desta forma, basta tomar $\mu$, $\sigma^2$ e $\nu$ tais que o logaritmo do lado direito em \eqref{eq:dist_post} seja maior do que $-745$.

Para o presente trabalho, foram considerados uma amostra de tamanho $n=500$ (Figura \ref{fig:sample_n}) da mistura finita de normais com variância contaminada parametrizada de tal forma que $\mu = 11$; $\sigma^2 = 0.64$ e $\nu = 0.2$, com hiperparâmetros $m = 11$; $V = 1$; $a = 7$ e $d = 4$ nas distribuições \textit{a priori}. Fixada uma semente aleatória, para a amostra gerada a partir dos valores citados, o logaritmo do núcleo da densidade \textit{a posteriori} é igual a $-518.9061$, cuja exponencial é aproximadamente igual a $4.385 \times 10^{-226}$.

\begin{figure}[htb]
	\centering
	\includegraphics[scale=0.8]{figuras/amostra_n.pdf}
	\caption{Histograma da amostra gerada do modelo, com média 11.1252; variância 14.3878; assimetria 0.7328 e curtose 13.7732}
	\label{fig:sample_n}
\end{figure}

Como não se tem uma expressão fechada para $p(\mu, \sigma^2, \nu | \bm{x})$, mas apenas de seu núcleo, para obter as densidades \textit{a posteriori} marginais de $\mu$, $\sigma^2$ e $\nu$ dado $\bm{x}$, bem como as estatísticas associadas a cada uma delas, é necessário aproximá-las por algum método numérico. Na seções \ref{quarie}, é apresentado o primeiro dos três métodos considerados: a integração via quadratura de Riemann. Por sua vez, na seção \ref{sir} é descrito o método SIR. Por fim, na seção \ref{mcmc}, é retratada a integração via MCMC com inovações MH. Todos os métodos serão testados e comparados no contexto do modelo exposto acima. Na seção \ref{consfin}, são resumidos os resultados obtidos para os três métodos, bem como é feita uma avaliação comparativa da qualidade de sua convergência.

Antes de implementar cada um dos métodos numéricos, foram feitos três gráficos (Figuras \ref{fig:maspro_mu} a \ref{fig:maspro_nu}) do núcleo de $p(\mu, \sigma^2, \nu | \bm{x})$ quando se varia um dos três parâmetros e os outros dois são mantidos fixos nos respectivos valores verdadeiros pressupostos para o modelo que gerou a amostra. Tais gráficos permitem dizer, para cada parâmetro, quais são os intervalos do suporte correspondente que concentram praticamente toda a massa probabilística da densidade \textit{a posteriori} marginal associada, ainda que as densidades calculadas sejam válidas apenas quando os outros dois parâmetros são fixados em valores arbitrários. Esta informação será útil para definir os intervalos de integração na quadratura de Riemann e a matriz de covariância da distribuição proposta nos métodos SIR e MCMC-MH para aproximação das densidades \textit{a posteriori} marginais verdadeiras.

\begin{figure}[htb]%
	\centering
	\subfloat[$I_{\mu} = (10.85,11.13)$, dados $\sigma^2 = 0.64, \nu = 0.2$]{
		{
			\label{fig:maspro_mu}
			\includegraphics[scale=0.4]{figuras/maspro_mu.pdf}}}%
	\qquad
	\subfloat[$I_{\sigma^2} = (0.48, 0.78)$, dados $\mu = 11, \nu = 0.2$]{
		{
			\label{fig:maspro_s2}
			\includegraphics[scale=0.4]{figuras/maspro_s2.pdf}}}%
	\subfloat[$I_{\nu} = (0.13, 0.26)$ dados $\mu = 11, \sigma^2 = 0.64$]{
		{
			\label{fig:maspro_nu}
			\includegraphics[scale=0.4]{figuras/maspro_nu.pdf}}}%
	\caption{Intervalos de massa probabilística para cada parâmetro (variável aleatória) do núcleo de $p(\mu, \sigma^2, \nu | \bm{x})$}%
\end{figure}
\newpage
\section{Quadratura de Riemann}\label{quarie}

Em problemas multidimensionais, a densidade marginal correspondente a cada variável aleatória para a qual se deseja fazer inferência é obtida integrando a densidade conjunta sobre as demais. Frequentemente, o cálculo das integrais pode ser difícil ou mesmo impossível analiticamente. No caso de uma densidade \textit{a posteriori} conjunta deve-se aproximar algumas das, ou todas as, integrais associadas a cada densidade \textit{a posteriori} marginal de interesse por métodos numéricos, os quais podem ser determinísticos ou estocásticos. Neste presente trabalho, isto será feito para todos os 3 parâmetros do modelo de mistura finita de normais com variância contaminada.

Dentre os métodos determinísticos mais usuais, para a aproximação das integrais correspondentes às densidades \textit{a posteriori} marginais de $\mu$, $\sigma^2$ e $\nu$, será utilizado o método da quadratura de Riemann, um caso particular e o mais simples da família das \textit{fórmulas de Newton-Cotes} (Chapra, 2015)\cite{Chapra2015}, as quais substituem o verdadeiro integrando por uma aproximação polinomial dentro de cada subintervalo contido no intervalo de integração. Na quadratura de Riemann, o integrando avaliado em cada subintervalo é aproximado por uma função constante avaliada no limite inferior (ou superior) do mesmo subintervalo.

Aproximações polinomiais com nós igualmente espaçados a cada subintervalo, como as regras de quadratura do Trapézio (grau 1) ou de Simpson (grau 2), obtém resultados com menor erro do que a quadratura de Riemann. Quando o grau da aproximação polinomial tende ao infinito, a quadratura converge para o verdadeiro valor da integral. Porém, em problemas de dimensão igual a 2 ou superior, usar aproximações polinomiais com nós torna-se muito custoso computacionalmente, uma vez que temos três ou mais termos na soma que aproximará a integral para cada dimensão. Por esta razão, escolheu-se a quadratura de Riemann para obter densidades \textit{a posteriori} marginais de $\mu$, $\sigma^2$ e $\nu$, pois esta regra possui um único termo na soma aproximadora, facilitando a programação das rotinas iterativas associadas. Em uma dada aproximação, o mesmo número de intervalos será usado para todas as dimensões.

Antes de aproximar as densidades \textit{a posteriori} marginais de cada parâmetro, é necessário aproximar o inverso da constante de proporcionalidade. Dados três parâmetros $(\alpha_1, \alpha_2, \alpha_3)$ e uma amostra dos dados $\bm{y}$ quaisquer, suponha que se deseja aproximar a densidade \textit{a posteriori} marginal de $\alpha_3$ dados os pontos $r_i, s_j, t_k$ da grade formada por todos os subintervalos de integração, $i, j, k \in \{1, \ldots, L\}$. Temos pela quadratura de Riemann que
\begin{align}
p(\alpha_3 | \bm{y})
&= \iint p(\alpha_1, \alpha_2, \alpha_3 | \bm{y}) d\alpha_1 d\alpha_2 \nonumber \\
\Rightarrow p(t_k | \bm{y})
&= \iint p(\alpha_1, \alpha_2, t_k | \bm{y}) d\alpha_1 d\alpha_2 \approx \sum_{i=1}^{L} \sum_{j=1}^{L} p(r_i, s_j, t_k | \bm{y}) \Delta_i \Delta_j \nonumber \\
&= \sum_{i=1}^{L} \sum_{j=1}^{L} c \cdot h(r_i, s_j, t_k | \bm{y}) \Delta_i \Delta_j. \label{eq:dpm_riem}
\end{align}

Como $c$, a constante de proporcionalidade, é dada pelo inverso da densidade \textit{a priori} preditiva $f(\bm{y})$, a qual é obtida integrando-se em todo o espaço paramétrico o produto entre a função de verossimilhança $f(\bm{y} | \alpha_1, \alpha_2, \alpha_3)$ e as densidades (ou funções de probabilidade) \textit{a priori} para $\alpha_1$, $\alpha_2$ e $\alpha_3$, também é possível aproximar $c$ pela quadratura de Riemann. Neste caso, $c^{-1} \approx \sum_{i=1}^{L} \sum_{j=1}^{L} \sum_{k=1}^{L} h(r_i, s_j, t_k | \bm{y}) \Delta_i \Delta_j \Delta_k$. Com o valor aproximado para $c$, é possível calcular \eqref{eq:dpm_riem} nos limites superior e inferior de todos os subintervalos de um dado parâmetro e enfim obter uma aproximação da densidade \emph{a posteriori} marginal deste mesmo parâmetro através de uma curva gráfica que liga todos os valores calculados. Evidentemente, quanto menor o tamanho de cada subintervalo, melhor a curva traçada aproximará a verdadeira densidade.

Para o modelo de mistura finita de normais com variância contaminada foram consideradas três quantidades distintas de subintervalos, correspondentes aos 3 cenários para a aproximação via quadratura de Riemann: $L = \{15, 50, 100\}$. Em cada cenário, dividiram-se os três intervalos de massa probabilística obtidos nas figuras \ref{fig:maspro_mu} a \ref{fig:maspro_nu} pelo respectivo valor de $L$. Nos três cenários, os valores calculados para o inverso da constante de proporcionalidade foram bem próximos: $3.9071 \times 10^{-229}$ para $L = 15$; $3.9014 \times 10^{-229}$ para $L = 50$ e $3.8999 \times 10^{-229}$ para $L = 100$. Com tais valores de $c^{-1}$, foram estimadas as densidades \textit{a posteriori} marginais para $\mu, \sigma^2$ e $\nu$, cujas curvas são apresentadas nas figuras \ref{fig:dpm_mu_qr_15} a \ref{fig:dpm_nu_qr_15}; \ref{fig:dpm_mu_qr_50} a \ref{fig:dpm_nu_qr_50} e \ref{fig:dpm_mu_qr_100} a \ref{fig:dpm_nu_qr_100}:

\begin{figure}[t]%
	\centering
	\subfloat[Densidade \textit{a posteriori} de $\mu$]{{
			\label{fig:dpm_mu_qr_15}
			\includegraphics[scale=0.4]{figuras/dpm_mu_qr_15.pdf}}}%
	\qquad
	\subfloat[Densidade \textit{a posteriori} de $\sigma^2$]{{
			\label{fig:dpm_s2_qr_15}
			\includegraphics[scale=0.4]{figuras/dpm_s2_qr_15.pdf}}}%
	\subfloat[Densidade \textit{a posteriori} de $\nu$]{{
			\label{fig:dpm_nu_qr_15}
			\includegraphics[scale=0.4]{figuras/dpm_nu_qr_15.pdf}}}%
	\caption{Densidades \textit{a posteriori} marginais pela quadratura de Riemann com $L = 15$}%
\end{figure}

\begin{figure}[t]%
	\centering
	\subfloat[Densidade \textit{a posteriori} de $\mu$]{{
			\label{fig:dpm_mu_qr_50}
			\includegraphics[scale=0.4]{figuras/dpm_mu_qr_50.pdf}}}%
	\qquad
	\subfloat[Densidade \textit{a posteriori} de $\sigma^2$]{{
			\label{fig:dpm_s2_qr_50}
			\includegraphics[scale=0.4]{figuras/dpm_s2_qr_50.pdf}}}%
	\subfloat[Densidade \textit{a posteriori} de $\nu$]{{
			\label{fig:dpm_nu_qr_50}
			\includegraphics[scale=0.4]{figuras/dpm_nu_qr_50.pdf}}}%
	\caption{Densidades \textit{a posteriori} marginais pela quadratura de Riemann com $L = 50$}%
\end{figure}

\begin{figure}[t]%
	\centering
	\subfloat[Densidade \textit{a posteriori} de $\mu$]{{
			\label{fig:dpm_mu_qr_100}
			\includegraphics[scale=0.4]{figuras/dpm_mu_qr_100.pdf}}}%
	\qquad
	\subfloat[Densidade \textit{a posteriori} de $\sigma^2$]{{
			\label{fig:dpm_s2_qr_100}
			\includegraphics[scale=0.4]{figuras/dpm_s2_qr_100.pdf}}}%
	\subfloat[Densidade \textit{a posteriori} de $\nu$]{{
			\label{fig:dpm_nu_qr_100}
			\includegraphics[scale=0.4]{figuras/dpm_nu_qr_100.pdf}}}%
	\caption{Densidades \textit{a posteriori} marginais pela quadratura de Riemann com $L = 100$}%
\end{figure}

Para $L=15$, a aproximação não é muito suave, mas começa a indicar como cada parâmetro se comporta com relação à locação e à dispersão de sua densidade \textit{a posteriori} marginal. Entretanto, ainda não é possível dizer como é o comportamento com relação à simetria.

Para $L=50$, a aproximação é bem suave. Além do comportamento com relação à locação e à dispersão, é possível dizer também como é o comportamento com relação à simetria da densidade \textit{a posteriori} marginal de cada parâmetro. Aparentemente, as densidades para $\sigma^2$ e $\nu$ são fracamente assimétricas à direita.

Para $L=100$, os gráficos confirma a tendência apresentada pelos dois cenários anteriores, mas com uma suavização ainda melhor. Porém, computacionalmente a integração numérica é bem mais demorada, uma vez que o número de pontos para cálculo das densidades \textit{a posteriori} marginais cresce a um fator de ordem $\mathcal{O}(L^3)$.

Obtidas as densidades \textit{a posteriori} marginais, é possível ainda estimar pela quadratura de Riemann os momentos \textit{a posteriori} de ordem $m$ para cada parâmetro e com os mesmos aproximar as estatísticas de média, variância, assimetria e curtose. Sem perda de generalidade, suponha que se deseja obter as estatísticas \textit{a posteriori} para $\alpha_1$. A aproximação pela quadratura de Riemann para um momento $m$ é dada por
\begin{align}
\mathbb{E}(\alpha_1^m | \bm{y}) &= \iint \alpha_1^m p(\alpha_1, \alpha_2, \alpha_3 | \bm{y}) d\alpha_1 d\alpha_2 d\alpha_3 \approx \sum_{i=1}^{L} \sum_{j=1}^{L} \sum_{k=1}^{L} r_i^m p(r_i, s_j, t_k | \bm{y}) \Delta_i \Delta_j \Delta_k \nonumber \\
&= \sum_{i=1}^{L} r_i^m \Delta_i \left[\sum_{j=1}^{L} \sum_{k=1}^{L} p(r_i, s_j, t_k | \bm{y}) \Delta_j \Delta_k\right] \nonumber \\
&= \sum_{i=1}^{L} r_i^m \Delta_i p(r_i | \bm{y}). \label{eq:mom_qr}
\end{align}

Como todas as estatísticas de interesse são funções dos quatro primeiros momentos, basta substituir $m = \{1, 2, 3, 4\}$ em \eqref{eq:mom_qr} para obter as respectivas aproximações. Na Tabela \ref{tab1}, são apresentados os valores aproximados para tais estatísticas em cada um dos três cenários.

\begin{table}[htb]
	\caption{Estatísticas \textit{a posteriori} para $(\mu, \sigma^2, \nu)$ pela quadratura de Riemann}
	\label{tab1}
	\centering
	\begin{tabular}{cccccc}
		\toprule
		Cenário & Parâmetro & Média & Variância & Assimetria & Curtose \\
		\midrule
		$L = 15$ & $\mu$      & 10.9847 & 0.0017 & 0.0022 & 2.9603 \\
				 & $\sigma^2$ &  0.6222 & 0.0022 & 0.2230 & 2.9984 \\
				 & $\nu$      &  0.1918 & 0.0004 & 0.1572 & 2.9348 \\
		\midrule
		$L = 50$ & $\mu$      & 10.9848 & 0.0017 & 0.0056 & 2.9346 \\
		         & $\sigma^2$ &  0.6222 & 0.0021 & 0.2149 & 2.9625 \\
		         & $\nu$      &  0.1918 & 0.0004 & 0.1523 & 2.8995 \\
		\midrule
		$L = 100$ & $\mu$      & 10.9848 & 0.0017 & 0.0064 & 2.9284 \\
		          & $\sigma^2$ &  0.6222 & 0.0021 & 0.2131 & 2.9542 \\
		          & $\nu$      &  0.1918 & 0.0004 & 0.1512 & 2.8913 \\
		\bottomrule
	\end{tabular}
\end{table}

Pela Tabela \ref{tab1}, pode-se concluir que tanto a média quanto a variância \textit{a posteriori} praticamente não variaram nos três cenários, independente do parâmetro considerado. Isto quer dizer que poucos subintervalos são necessários para se obter uma boa aproximação destas duas estatísticas. Por outro lado, o mesmo não pode ser dito para a assimetria e curtose \textit{a posteriori}, para as quais há mudanças na terceira ou mesmo na segunda casa decimal, inclusive quando se passa de $L = 50$ para $L = 100$ subintervalos. Isto se explica pelo fato de que tanto a assimetria quanto a curtose são funções que dependem de vários momentos, até a terceira e quarta ordem respectivamente. Portanto, há maior dificuldade em aproximá-las. Apesar deste fato, para o problema considerado se pode dizer que houve uma estabilidade nas aproximações destas duas últimas estatísticas na medida em que a quantidade de subintervalos crescia.

Com relação aos valores em si, as médias \textit{a posteriori} para os 3 parâmetros ficaram bem próximas, mas não necessariamente iguais, aos respectivos valores do modelo para a distribuição amostral, mesmo para uma amostra muito grande ($n = 500$). É interessante notar que isso ocorre mesmo para o parâmetro $\nu$, cuja distribuição \textit{a priori} é não-informativa no sentido de Bayes-Laplace. Apesar disto, sua distribuição \textit{a posteriori} é influenciada pelas informações de $\mu$ e $\sigma^2$ contidas tanto nas distribuições \textit{a priori} correspondentes quanto na função de verossimilhança.

Para as variâncias \textit{a posteriori}, todas elas foram bem pequenas, em especial para $\nu$. Isto se deve ao fato de que a amostra inicial dos dados era grande, portanto se poderia esperar uma redução na dispersão de cada parâmetro quando sua informação é atualizada com a distribuição amostral, por menos informativa que fosse a distribuição \textit{a priori} (como no caso de $\nu$). Com relação à assimetria \textit{a posteriori}, esta foi praticamente nula para $\mu$ e positiva, mas fraca, para $\sigma^2$ e $\nu$ (um pouco mais forte para a primeira). Por fim, as aproximações para a curtose \textit{a posteriori} foram todas bem próximas de 3, o valor encontrado para uma distribuição normal, com o parâmetro $\sigma^2$ mais próximo desse valor e $\nu$ o mais afastado.

\section{O Método da Reamostragem Por Importância Sequencial (SIR)}\label{sir}

%Texto descrevendo o método e como foi feita neste trabalho a geração de variáveis aleatórias da distribuição de interesse usando este mesmo método.

\section{O Método de Monte Carlo via Cadeias de Markov (MCMC)}\label{mcmc}

%Texto descrevendo o método e como foi feita neste trabalho a geração de variáveis aleatórias da distribuição de interesse usando este mesmo método, bem como o amostrador selecionado (possivelmente Gibbs ou outro que for de melhor uso para todo o grupo).

\section{Considerações Finais}\label{consfin}

%Texto descrevendo de forma resumida o que foi feito nas seções anteriores, bem como os resultados dos métodos quanto à convergência, à medida que se aumenta o tamanho amostral m da distribuição \textit{a poseteriori}.

\bibliographystyle{unsrt}
\bibliography{ref}

\newpage

\section*{Apêndice: Código R para os Resultados Apresentados}

\subsection*{Introdução}

\begin{verbatim}
# Tamanho amostral "n"; parâmetros "mu", "s2" (sigma^2) e "nu" do modelo para
# a distribuição amostral e hiperparâmetros m, V, a e d das distribuições "a
# priori" (somente para "mu" e "s2"):

n = 500; mu = 11; s2 = 0.64; nu = 0.2; m = 11; V = 1; a = 7; d = 4

# Geração hierárquica da amostra de tamanho "n" do modelo:

U_pdf = function(n, nu) {
	u = sample(c(100, 1), prob=c(nu, 1-nu), size=n, replace=T)
}

set.seed(122019)

u = U_pdf(n, nu)
sam = rnorm(n, mu, sqrt(s2*u))

# Histograma da amostra gerada do modelo (figura 1):

par(mar = c(5,5,3,2))
hist(sam, breaks=100, prob=T, main="", xlim = c(-10, 40), ylim = c(0, 0.5),
	 xlab = "Amostra do modelo", ylab = "Densidade empírica")

# Cálculo do logaritmo do núcleo da distribuição "a posteriori" original:

logA = function(X, mu, sigma2, nu) {
	n = length(X)
	k1 = (nu/10)*exp(-(X-mu)^2/(2*100*sigma2))
	k2 = (1-nu)*exp(-(X-mu)^2/(2*sigma2))
	k = log(k1 + k2)
	return(k)
}

logh = function(n, mu, sigma2, nu, m, V, a, d) {
	k1 = -((n + 1)/2 + a + 1)*log(sigma2)
	k2 = -((mu - m)^2/(2*V) + d)/sigma2
	k = k1 + k2
	return(k)
}

logkpost = function(X, mu, sigma2, nu, m, V, a, d) {
	n = length(X)
	lA = logA(X, mu, sigma2, nu)
	lh = logh(n, mu, sigma2, nu, m, V, a, d)
	lkp = sum(lA) + lh
	return(lkp)
}

# Gráficos dos intervalos de massa probabilística para cada parâmetro do
# núcleo da distribuição "a posteriori", fixados os demais, após algumas
# tentativas anteriores para redução dos limites do gráfico e melhor vi-
# sualização da curva (figuras 2a a 2c):

mu_sup=seq(10.8, 11.2, 0.001); t_mu=length(mu_sup); kp_mu=numeric(t_mu)

for(i in 1:t_mu) {
	kp_mu[i] = exp(logkpost(X=sam,
	mu=mu_sup[i], sigma2=s2, nu=nu, m=m, V=V, a=a, d=d))
	}
plot(mu_sup, kp_mu, type="l", main="",
     xlab=expression(paste(mu)), ylab="")

s2_sup=seq(0.4, 0.8, 0.001); t_s2=length(s2_sup); kp_s2=numeric(t_s2)

for(i in 1:t_s2) {
	kp_s2[i] = exp(logkpost(X=sam,
	mu=mu, sigma2=s2_sup[i], nu=nu, m=m, V=V, a=a, d=d))
	}
plot(s2_sup, kp_s2, type="l", main="",
	 xlab=expression(paste(sigma^2)), ylab="")

nu_sup=seq(0.1, 0.3, 0.001); t_nu=length(nu_sup); kp_nu=numeric(t_nu)

for(i in 1:t_nu) {
	kp_nu[i] = exp(logkpost(X=sam,
	mu=mu, sigma2=s2, nu=nu_sup[i], m=m, V=V, a=a, d=d))
	}
plot(nu_sup, kp_nu, type="l", main="",
     xlab=expression(paste(nu)), ylab="")
\end{verbatim}

\subsection*{O Método da Quadratura de Riemann}

\begin{verbatim}

# Cenários considerados, variando no número de subintervalos utilizados:

L1 = 15; L2 = 50; L3 = 100

# Tamanhos dos subintervalos, dada a combinação de cenário e parâmetro, e
# grades definidas pelos mesmos:

mu_step1 = (11.13 - 10.85)/L1; mu_gr1 = seq(10.85, 11.13, mu_step1)
mu_step2 = (11.13 - 10.85)/L2; mu_gr2 = seq(10.85, 11.13, mu_step2)
mu_step3 = (11.13 - 10.85)/L3; mu_gr3 = seq(10.85, 11.13, mu_step3)

s2_step1 = (0.78 - 0.48)/L1; s2_gr1 = seq(0.48, 0.78, s2_step1)
s2_step2 = (0.78 - 0.48)/L2; s2_gr2 = seq(0.48, 0.78, s2_step2)
s2_step3 = (0.78 - 0.48)/L3; s2_gr3 = seq(0.48, 0.78, s2_step3)

nu_step1 = (0.26 - 0.13)/L1; nu_gr1 = seq(0.13, 0.26, nu_step1)
nu_step2 = (0.26 - 0.13)/L2; nu_gr2 = seq(0.13, 0.26, nu_step2)
nu_step3 = (0.26 - 0.13)/L3; nu_gr3 = seq(0.13, 0.26, nu_step3)

# Combinando as três grades unidimensionais em cada cenário, todas com
# número igual de subintervalos e de pontos (nos limites dos interval-
# os) para integração numérica:

grid_tri1 = cbind(mu_gr1, s2_gr1, nu_gr1); l1 = nrow(grid_tri1)
grid_tri2 = cbind(mu_gr2, s2_gr2, nu_gr2); l2 = nrow(grid_tri2)
grid_tri3 = cbind(mu_gr3, s2_gr3, nu_gr3); l3 = nrow(grid_tri3)

# Produto triplo dos tamanhos dos subintervalos em cada cenário:

prod_step1 = mu_step1*s2_step1*nu_step1
prod_step2 = mu_step2*s2_step2*nu_step2
prod_step3 = mu_step3*s2_step3*nu_step3

# Cálculo do inverso da constante de proporcionalidade em cada cenário:

cprop = function(l, X, mgr, s2gr, ngr, prst, m, V, a, d) {
	c = 0
	for (i in 1:l) {
		for (j in 1:l) {
			for (k in 1:l) {
				aux = logkpost(X, mgr[i], s2gr[j], ngr[k], m, V, a, d)
				c = c + exp(aux)*prst
			}
		}
	}
	return(c)
}

c1 = cprop(l=l1, X=sam, mgr=mu_gr1, s2gr=s2_gr1, ngr=nu_gr1,
		   prst=prod_step1, m=m, V=V, a=a, d=d)
c2 = cprop(l=l2, X=sam, mgr=mu_gr2, s2gr=s2_gr2, ngr=nu_gr2,
		   prst=prod_step2, m=m, V=V, a=a, d=d)
c3 = cprop(l=l3, X=sam, mgr=mu_gr3, s2gr=s2_gr3, ngr=nu_gr3,
		   prst=prod_step3, m=m, V=V, a=a, d=d)
c1; c2; c3

# Cálculo das densidades "a posteriori" marginais em cada cenário, com
# uso do produto duplo dos tamanhos dos subintervalos em cada cenário:

pr1_dup12 = mu_step1*s2_step1
pr1_dup13 = mu_step1*nu_step1
pr1_dup23 = s2_step1*nu_step1
\end{verbatim}

\newpage

\begin{verbatim}
pr2_dup12 = mu_step2*s2_step2
pr2_dup13 = mu_step2*nu_step2
pr2_dup23 = s2_step2*nu_step2

pr3_dup12 = mu_step3*s2_step3
pr3_dup13 = mu_step3*nu_step3
pr3_dup23 = s2_step3*nu_step3

postmu_quarie = function(l, X, mgr, s2gr, ngr, prst, m, V, a, d, c) {
	postmu = numeric(l)
	for (i in 1:l) {
		postconj = 0
		for (j in 1:l) {
			for(k in 1:l) {
				aux = logkpost(X, mgr[i], s2gr[j], ngr[k], m, V, a, d)
				postconj = postconj + exp(aux)*prst
			}
		}
		postmu[i] = postconj/c
	}
	return(postmu)
}

pmq1 = postmu_quarie(l=l1, X=sam, mgr=mu_gr1, s2gr=s2_gr1, ngr=nu_gr1,
					 prst=pr1_dup23, m=m, V=V, a=a, d=d, c=c1)
pmq2 = postmu_quarie(l=l2, X=sam, mgr=mu_gr2, s2gr=s2_gr2, ngr=nu_gr2,
					 prst=pr2_dup23, m=m, V=V, a=a, d=d, c=c2)
pmq3 = postmu_quarie(l=l3, X=sam, mgr=mu_gr3, s2gr=s2_gr3, ngr=nu_gr3,
					 prst=pr3_dup23, m=m, V=V, a=a, d=d, c=c3)

plot(mu_gr1,pmq1,type="l",main="",xlab=expression(paste(mu)),ylab="")
plot(mu_gr2,pmq2,type="l",main="",xlab=expression(paste(mu)),ylab="")
plot(mu_gr3,pmq3,type="l",main="",xlab=expression(paste(mu)),ylab="")

posts2_quarie = function(l, X, mgr, s2gr, ngr, prst, m, V, a, d, c) {
	posts2 = numeric(l)
	for (i in 1:l) {
		postconj = 0
		for (j in 1:l) {
			for(k in 1:l) {
				aux = logkpost(X, mgr[j], s2gr[i], ngr[k], m, V, a, d)
				postconj = postconj + exp(aux)*prst
			}
		}
		posts2[i] = postconj/c
	}
	return(posts2)
}

\end{verbatim}

\newpage

\begin{verbatim}
psq1 = posts2_quarie(l=l1, X=sam, mgr=mu_gr1, s2gr=s2_gr1, ngr=nu_gr1,
					 prst=pr1_dup13, m=m, V=V, a=a, d=d, c=c1)
psq2 = posts2_quarie(l=l2, X=sam, mgr=mu_gr2, s2gr=s2_gr2, ngr=nu_gr2,
					 prst=pr2_dup13, m=m, V=V, a=a, d=d, c=c2)
psq3 = posts2_quarie(l=l3, X=sam, mgr=mu_gr3, s2gr=s2_gr3, ngr=nu_gr3,
					 prst=pr3_dup13, m=m, V=V, a=a, d=d, c=c3)

plot(s2_gr1, psq1, type="l", main="", xlab=expression(paste(sigma^2)),
	 ylab="")
plot(s2_gr2, psq2, type="l", main="", xlab=expression(paste(sigma^2)),
	 ylab="")
plot(s2_gr3, psq3, type="l", main="", xlab=expression(paste(sigma^2)), 
	 ylab="")

postnu_quarie = function(l, X, mgr, s2gr, ngr, prst, m, V, a, d, c) {
	postnu = numeric(l)
	for (i in 1:l) {
		postconj = 0
		for (j in 1:l) {
			for(k in 1:l) {
				aux = logkpost(X, mgr[j], s2gr[k], ngr[i], m, V, a, d)
				postconj = postconj + exp(aux)*prst
			}
		}
		postnu[i] = postconj/c
	}
	return(postnu)
}

pnq1 = postnu_quarie(l=l1, X=sam, mgr=mu_gr1, s2gr=s2_gr1, ngr=nu_gr1,
					 prst=pr1_dup12, m=m, V=V, a=a, d=d, c=c1)
pnq2 = postnu_quarie(l=l2, X=sam, mgr=mu_gr2, s2gr=s2_gr2, ngr=nu_gr2,
					 prst=pr2_dup12, m=m, V=V, a=a, d=d, c=c2)
pnq3 = postnu_quarie(l=l3, X=sam, mgr=mu_gr3, s2gr=s2_gr3, ngr=nu_gr3,
					 prst=pr3_dup12, m=m, V=V, a=a, d=d, c=c3)

plot(nu_gr1,pnq1,type="l",main="",xlab=expression(paste(nu)),ylab="")
plot(nu_gr2,pnq2,type="l",main="",xlab=expression(paste(nu)),ylab="")
plot(nu_gr3,pnq3,type="l",main="",xlab=expression(paste(nu)),ylab="")

# Cálculo da média, variância, assimetria e curtose "a posteriori" para
# cada parâmetro:

stat_post = function(gr, marg, prst) {
	media = 0
	var = 0
	assim = 0
	cur = 0
	l = length(gr)
	
	aux1 = sum(gr*marg*prst)     # Aproxima 1º momento.
	aux2 = sum((gr^2)*marg*prst) # Aproxima 2º momento.
	aux3 = sum((gr^3)*marg*prst) # Aproxima 3º momento.
	aux4 = sum((gr^4)*marg*prst) # Aproxima 4º momento.
	
	media = aux1
	var = aux2 - (media)^2
	assim = (aux3 - 3*media*var - media^3)/(var^(3/2))
	cur = (aux4 - 4*media*aux3 + 6*media^2*aux2 - 3*(media^4))/(var^2)
	return(list(media, var, assim, cur))
}

stat_post(mu_gr1, pmq1, mu_step1)
stat_post(mu_gr2, pmq2, mu_step2)
stat_post(mu_gr3, pmq3, mu_step3)

stat_post(s2_gr1, psq1, s2_step1)
stat_post(s2_gr2, psq2, s2_step2)
stat_post(s2_gr3, psq3, s2_step3)

stat_post(nu_gr1, pnq1, nu_step1)
stat_post(nu_gr2, pnq2, nu_step2)
stat_post(nu_gr3, pnq3, nu_step3)

\end{verbatim}

\subsection*{O Método da Reamostragem Por Importância Sequencial (SIR)}

\begin{verbatim}
a
\end{verbatim}

\subsection*{O Método de Monte Carlo via Cadeias de Markov (MCMC)}

\begin{verbatim}
a
\end{verbatim}

\end{document}