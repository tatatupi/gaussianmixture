\section{Considerações Finais}\label{consfin}

%Texto descrevendo de forma resumida o que foi feito nas seções anteriores, bem como os resultados dos métodos quanto à convergência, à medida que se aumenta o tamanho amostral m da distribuição \textit{a poseteriori}.

Neste trabalho, as densidades marginais \textit{a posteriori} dos parâmetros de interesse ($\mu, \sigma^2, \nu$) foram aproximadas por três métodos numéricos, para uma amostra observada de tamanho 500. A distribuição \textit{a priori} marginal para o parâmetro $\mu$ foi escolhida com a média igual ao valor real (11) e variância igual a $V\sigma^2 = 0.64$. Para $\sigma^2$, a gama inversa escolhida possui uma média igual a $4$/$6$ (aproximadamente 0,67), próxima do valor real de 0.64, com variância aproximadamente igual a 0,089. Por fim, para o parâmetro $\nu$, a distribuição \textit{a priori} uniformemente distribuida entre 0 e 1 foi a única não-informativa, no sentido Bayes-Laplace, para um valor real igual a 0.2. Logo, esperava-se que as médias aproximadas das distribuições marginais \textit{a posteriori} fossem próximas dos valores reais dos parâmetros da distribuição amostral. De fato, os resultados apresentados nas Tabelas \ref{tab1}, \ref{tab2} e \ref{tab3} confirmaram as expectativas para os três métodos.

Para os três métodos implementados, foram desenvolvidos 3 cenários diferentes, para análise de convergência e complexidade computacional. No primeiro caso, do método de quadratura, três quantidades diferentes de intervalos de integração $L$ foram escolhidos: 15, 50 e 100. O primeiro resultado não é muito satisfatório, apesar de aproximar bem os dois primeiros momentos. Para 50 e 100 subintervalos, a aproximação é bem melhor, às custas de um custo computacional maior, crescendo exponencialmente em relação a dimensão do espaço paramétrico ($O(L^n)$), com $n$ igual a 3. Para os métodos estocásticos SIR e MCMC, foram considerados, respectivamente, tamanhos amostrais e quantidade de iterações $k$ iguais a 500, 5000 e 50000. Em ambos os casos, para $k$ igual a 500, as distribuições \textit{a posteriori} aproximadas não foram muito satisfatórias, apresentando muitos \textit{outliers} e modas locais. Para o SIR, a partir de 5000 amostras, as distribuições são bem mais suaves, sendo possível observar o comportamentos de momentos de ordem superior. No MCMC, ainda se observa uma melhoria considerável de $k$ igual a 5000 para $k$ igual a 50000 passos de integração, quando o método parece convergir.

Por fim podemos concluir que para o modelo probabilístico estudado todos os três métodos desempenham bem a tarefa de aproximar a distribuição \textit{a posteriori}, se escolhidos hiperparâmetros de convergência adequados. No entanto, para dimensões paramétricas maiores, apenas os métodos estocásticos são passíveis de serem utilizados, já que a ordem de complexidade cresce exponencialmente em relação a $n$. Entre os métodos estocásticos, o método SIR apresentou um resultado melhor, precisando de menos amostras para produzir uma aproximação satisfatória. Isso era esperado, uma vez que o MCMC possui desvantagens, como por exemplo a correlação entre as amostras. No entanto, à medida que a dimensão do espaço amostral aumenta, o SIR pode sofrer com uma baixa taxa de aceitação das amostras (Tokdar, 2010)\cite{Tokdar2010}, fazendo com que o MCMC tenha melhores resultados para um mesmo valor de $k$.